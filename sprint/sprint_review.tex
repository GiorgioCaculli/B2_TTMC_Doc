Durant ce troisième et dernier sprint, quelques nouveautés ont fait leurs apparitions.\\
Un plateau de jeu a été ajouté en solo ainsi qu'en multi local.
Le multi joueur local accepte plus de deux joueurs ( contrairement à avant où deux était le maximum).

\begin{itemize}
	\item Objectif atteint
	\begin{itemize}
		\item Connaitre son score.
		\item Savoir si la réponse donnée était la bonne.
		\item Savoir si la réponse donnée était mauvaise.
		\item Savoir la bonne réponse si la réponse donnée était mauvaise.
		\item Possibilité de mettre le jeu en pause.
		\item Reprendre mon jeu là où je l'avais laissé.
		\item Possibilité de quitter mon jeu à tout môment.
		\item Présence d'un multi-joueur local.
		\item Possibilité d'ajouter une nouvelle carte.
		\item Possibilité de retirer une carte.
		\item Possibilité de modifier une carte.
		\item Présence d'une musique de fond.
		\item Utilisation de pseudo créé par l'utilisateur.
		\item Création et utilisation d'un plateau de jeu.
		\item Création de pion de couleur différente pour chaque joueur.
		\item Création d'un chat en multi joueur en ligne.
		\item Possibilité de régler le volume sonore.
		\item Possibilité de désactiver la musique de fond.
		\item Création d'un launcher permettant de télécharger les librairies nécessaires au fonctionnement du jeu. Il permet aussi de lancer le jeu.
	\end{itemize}
	\item Objectif non atteint
	\begin{itemize}
		\item Jeu en multi joueur en ligne opérationnel.
	\end{itemize}
\end{itemize}