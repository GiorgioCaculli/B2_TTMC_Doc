\noindent\adjustbox{max width=\textwidth}{
\begin{tabular}{| p{1cm} | p{16cm} |}
	\hline
	US-01 & En tant qu'utilisateur je dois pouvoir lancer la partie.\\
	\hline
	US-02 & En tant qu'utilisateur je voudrais savoir mon score.\\
	\hline
	US-03 & En tant qu'utilisateur je voudrais savoir si j'ai bien répondu.\\
	\hline
	US-04 & En tant qu'utilisateur je voudrais savoir si j'ai mal répondu.\\
	\hline
	US-05 & En tant qu'utilisateur je voudrais savoir quel était la bonne réponse.\\
	\hline
	US-06 & En tant qu'utilisateur je voudrais savoir mettre mon jeu sur pause.\\
	\hline
	US-07 & En tant qu'utilisateur je voudrais savoir reprendre mon jeu où je l'avais laissé.\\
	\hline
	US-08 & En tant qu'utilisateur je voudrais savoir arreter mon jeu à tout moment.\\
	\hline
	US-09 & En tant qu'utilisateur j'aimerais joué en multi joueur localement.\\
	\hline
	US-10 & En tant qu'administrateur  je dois pouvoir ajouter une nouvelle carte au deck.\\
	\hline
	US-11 & En tant qu'administrateur je veux pouvoir supprimer une carte du deck.\\
	\hline
	US-12 & En tant qu'administrateur je veux pouvoir modifier une carte existante.\\
	\hline
	US-13 & En tant qu'utilisateur j'aimerais avoir une musique de fond.\\
	\hline
	US-14 & En tant qu'utilisateur j'aimerais pouvoir gérer le volume de la musique.\\
	\hline
	US-15 & En tant qu'utilisateur j'aimerais pouvoirs activé ou désactivé la musique de fond.\\
	\hline
	US-16 & En tant qu'utilisateur je voudrais pouvoir choisir mon propre pséudonyme.\\
	\hline
	US-17 & Création d'un plateau de jeu.\\
	\hline
	US-18 & implémentation du plateau de jeu.\\
	\hline
	US-19 & creation des pions.\\
	\hline
	US-20 & implémentation des pions de jeu et animation de mouvements des pions.\\
	\hline
	US-21 & Investiguer sur le multi joueur en ligne.\\
	\hline
\end{tabular}
}
