Dans le cadre du cours de ``Projets'' de l'UE 210, nous allons devoir créer un jeu similaire à ``\acrlong{ttmc}'' (\acrshort{ttmc}). Il s'agit d'un jeu comportant des cartes comportant des questions. Il existe quatre thèmes repérable par leurs couleurs respectives, la couleur:
\begin{itemize}
	\item mauve est attribué au carte ayant pour thème ``improbable''
	\item orange pour ``plaisir''
	\item bleu pour ``informatique''
	\item vert pour ``scolaire''
\end{itemize}
Chaque cartes possède un thème, un sujet en rapport avec le thème, et quatre questions.\\
Les question sont numéroté de un à quatre et trié par ordre croissant de difficulté.\\
Le jeu commence avec un joueur qui tire une carte, il pose la question "tu te mets combien en (sujet de la carte) ?" à l'autre joueur, l'autre joueur lui répond un chiffre entre un et quatre, le joueur ayant tiré la carte pose alors la question correspondante au nombre donné par l'autre joueur. En cas de bonne réponse, le joueur ayant répondu gagne un nombre de points équivalent au numéro de la question. En cas de mauvaise réponse, le joueur ne gagne aucun point. C'est ensuite au joueurs ayant répondu de tirer une carte et d'interroger l'autre.

