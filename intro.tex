Dans le cadre du cours de "Projets" de l'UE 210, nous allons devoir créé un jeu similaire à "Tu te mets combien ?" (ttmc). Il s'agit d'un jeu comportant des cartes comportant des questions. Il existe quatre thèmes répérable par leurs couleurs 
respéctives, la couleurs mauve est attribué au carte ayant pour thème "improbable", orange pour "plaisir", bleu pour "informatique", vert pour "scolaire". Pour en revenir au carte, chaque cartes possède un thème, un sujet en rapport avec le thème,
et quatre questions. Les question sont numéroté de un à quatre et trié par ordre croissant de difficulté. Le jeu commence avec un joueur qui tire une carte, il pose la question "tu te mets combien en (sujet de la carte) ?" à l'autre joueur. L'autre
joueur lui répond un chiffre entre un et quatre, le joueur ayant tiré la carte pose alors la question correspondante au nombre donné par l'autre joueur. En cas de bonne réponse, le joueur ayant répondu gagne un nombre de points équivalent au numéro 
de la question. En cas de mauvaise réponse, le joueur ne gagne aucun point. C'est ensuite au joueurs ayant répondu de tirer une carte et d'intéroger l'autre.
   
