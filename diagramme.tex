%\umlDiagram[box=,sizeX=7cm, sizeY=7cm]{
%	\umlClass[]{Deck}
%	{}
%	{
%		\umlMethod[visibility]{add}{c \emph{BasicCard}}
%		\umlMethod[visibility]{remove}{c \emph{BasicCard}}
%		\umlMethod[visibility]{remove}{i \emph{int}}
%	}
%}
\begin{figure}[h]
	\centering
	\includegraphics[width=\textwidth]{TTMC_Model_Diagram.png}
	\caption{Diagramme de classe du modèle}
	\label{fig:diag_modele}
\end{figure}

\begin{figure}[h]
	\centering
	\includegraphics[width=\textwidth]{DiagrammeClasseVue.png}
	\caption{Diagramme de classe du modèle}
	\label{fig:diag_modele}
\end{figure}

Nous avons décidé de travailler avec le MainGui en position centrale. Lorsque nous changeons de scène via le click d’un bouton, nous modifier la scène via le .setScene() directement dans le MainGui. Ce qui nous permet d’avoir une répartition centralisée des modifications de scènes.
