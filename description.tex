\subsection{Fichier}
Le jeu utilisera comme fichier de sauvegarde pour les cartes ainsi que pour le deck le format \acrshort{json}. Format très populaire et qui facilite grandement l'échange, le chargement et l'écriture des données. Le jeu intègre donc une bibliothèque 
de création et lecture du format \acrshort{json}.

\subsection{Spécification technique}
Le programme est développé entièrement dans le langage de programmation Java. Plus précisément, la version 11 du \acrshort{jdk}  (\acrlong{jdk}).\\
Les librairies utilisées seront:
\begin{itemize}
	\item JavaFX 11
	\item GSON 2
	\item JUnit 5
\end{itemize}
La plupart des fonctionnalités seront basés sur la plupart des librairies natives que Java propose, fonctionnalités comme:
\begin{itemize}
	\item Logger
	\item Net
	\item Thread
\end{itemize}

\subsection{L'interface graphique utilisateur (GUI)}
L'utilisateur aura comme principale interface graphique une fenêtre avec laquelle il pourra accéder aux différentes parties du programme. Tout affichage sur la fenêtre dépendra des interactions que l'utilisateurs aura avec les différents menus qui lui seront proposés. De manière générale, l'interface démarrera sur un menu principale qui mènera l'utilisateur vers les différentes partie du logiciel:
\begin{itemize}
	\item Le menu pour démarrer la partie
	\item Le menu pour paramétrer le programme
	\item Le menu pour manipuler les composants du jeu
\end{itemize}
Une variété de Panes lui seront proposés, et chaque Pane aura sa propre utilité et importance.
