Durant ce deuxième sprint,\\
Nous avons créé de nouvelles interactions entre l'utilisateur et le logiciel de jeu.
Nous avons dévellopé une interface pour afficher les cartes (visible depuis le menu administrateur).
Une nouvelle fonctionnalité concernant la gestion des cartes a également été programmée, il s'agit de la possibilité de supprimer une carte du deck.
Autre changement concernant la gestion des cartes, l'interface permettant de modifier les cartes est également faite.
Le jeu lui-même a également bénéficié d'ajout tel qu'un système de score.
Un multi joueur local à aussi été créé (pour le moment uniquement en 1V1).
Le jeu prend en charge les pseudo pour une partie multi joueur local tout comme pour une partie solo.
Un menu pause à également été créé, celui-ci permet de mettre en pause la partie.
Depuis ce menu, il est possible soit de reprendre la partie ou de la quitter.
Des animations et un timer ont été rajoutés afin de rendre le jeu plus intéressant et dynamique.
Afin de rendre le jeu agréable nous avons ajouté une playlist de musique ainsi que quelque fonctionnalité pour pouvoir gérer celle-ci.
Dernière modification concernant le jeu, l'interface de jeu prend comme couleur de fond la couleur du thème de la carte tirée (c'est à dire que lors du choix de la difficulté de la question ou lors d'y répondre, la couleur de fond de l'interface est colorée en rapport avec la carte tirée).
Des interactions clavier-logiciel ont été créées, notamment, il est dès lors possible de faire apparaitre le menu pause grace à la touche ESCAPE ainsi que d'utiliser la touche ENTER pour valider un choix.
Pour améliorer la sécurité du jeu, la programmation de la fenêtre de login est faite.
Nous avons décidé de créé un launcher pour notre jeu permettant aux utilisateurs n'ayant pas JavaFX de pouvoir quand même y jouer.\\
Une partie du travail fait durant ce sprint a également été de modifier quelque partie du code afin de satisfaire le \acrshort{po}.
Nous avons fait une conversion totale du code afin d'éviter d'utiliser plusieurs \textquote{Scene}, la nouvelle solution mise en place est l'utilisation d'une seule \textquote{Scene} combinée à plusieurs \textquote{StackPane}.  
