Durant ce deuxième sprint,\\
Nous avons créé de nouvelles interactions entre l'utilisateur et le logiciel de jeu.
Nous avons développé une interface pour afficher les cartes.
Une nouvelle fonctionnalité concernant la gestion des cartes a également été programmée, il s'agit de la possibilité de supprimer une carte du deck.
Autre changement concernant la gestion des cartes, l'interface permettant de modifier les cartes est également opérationnelle.
Les options présentées ci-dessus sont accéssible depuis le menu administrateur.
Le jeu lui-même a également bénéficié d'ajout tel qu'un système de score.
Un multi joueur local a aussi été créé (pour le moment uniquement en 1v1).
Le jeu prend en charge les pseudos pour une partie multi joueur locale tout comme pour une partie solo.
Un menu pause a également été créé, celui-ci permet de mettre en pause la partie.
Depuis ce menu, il est possible soit de reprendre la partie ou de la quitter.
Des animations et un timer ont été rajoutés afin de rendre le jeu plus intéressant et dynamique.
Afin de rendre le jeu agréable nous avons ajouté une playlist de musiques ainsi que quelques fonctionnalités pour pouvoir gérer celle-ci.
Dernière modification concernant le jeu, l'interface de jeu prend comme couleur de fond la couleur du thème de la carte tirée (c'est à dire, lors du choix de la difficulté de la question ou lors d'y répondre, la couleur de fond de l'interface est colorée en rapport avec le thème la carte tirée).
Des interactions clavier-logiciel ont été créées, il est dès lors possible de, notamment, faire apparaitre le menu pause grace à la touche ESCAPE ainsi que d'utiliser la touche ENTER pour valider un choix.
Pour améliorer la sécurité du jeu, la fenêtre de login a été mise en place.
Nous avons décidé de créer un launcher pour notre jeu permettant aux utilisateurs n'ayant pas JavaFX de pouvoir quand même y jouer.\\
Une partie du travail fait durant ce sprint a également été de modifier quelques parties du code afin de satisfaire le \acrshort{po}.
Nous avons fait une conversion totale du code afin d'éviter d'utiliser plusieurs Scenes, la nouvelle solution mise en place est l'utilisation d'une seule et unique Scene, combinée à plusieurs StackPanes.  
