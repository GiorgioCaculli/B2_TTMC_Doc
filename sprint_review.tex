Durant ce deuxième sprint,\\
Nous avons créé de nouvelle interactions entre l'utilisateur et le logiciel de jeux. Nous avons dévellopé une interface pour 
afficher les cartes (visible depuis le menu administrateur). Une nouvelle fonctionnalité concernant la gestion des cartes a également 
été programmer, il s'agit de la possibilité de supprimer une carte du deck. Autre changement concernant la gestion des cartes, 
l'interface permettant de modifier les cartes est également faites. Le jeu lui-même a également bénéficier d'ajout telle qu'un système 
de score. Un multi joueur local à aussi été créé (pour le moment uniquement en 1V1). Le jeu prend en charge les pseudo pour une partie 
multi joueur local tout comme pour une partie solo. Un menu pause à également été créé, celui-ci permet de mettre en pause la partie.
Depuis ce menu, il est possible soit de reprendre la partie ou de quitter le jeu. Des animation et un timer ont été rajouter afin de 
rendre le jeu plus intéressant et dynamique. Afin de rendre le jeu agréable nous avons ajouter une playlist de musique ainsi que 
quelque fonctionnalité pour pouvoir gérer celle-ci.Dernière modification concernant le jeu, l'interface de jeu prend comme couleur de 
fond la couleur du thème de la carte tiré ( c'est à dire que lors du choix de la difficulté de la question ou lors d'y répondre, la 
couleurde fond de l'interface est coloré en rapport avec la carte tiré). Des interaction clavier logiciel ont été créé, notamment, il 
est dès lors possible de faire apparaitre le menu pause grace à la touche escape ainsi que d'utiliser la touche "enter" pour valider 
un choix. Pour améliorer la sécurité du jeu, la programmation de la fenêtre de login est faite. Nous avons décider de créé un launcher 
pour notre jeu permettant au utilisateur n'ayant pas JavaFX de pouvoir quand même y joué.\\
Une partie du travail fait durant se sprint 2 a également été de modifier quelque partie du code afin de satisfaire le PO.
Nous avons fait une conversion total du code afin d'éviter d'utiliser plusieurs "scène", la nouvelle solution mise en place est
l'utilisation d'une seule "scène" conbiné à plusieurs Stackpanes.  