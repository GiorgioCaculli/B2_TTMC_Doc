Durant ce premier sprint, concernant le code,
\begin{itemize}
	\item Les classes Questions, Themes, Deck et BasicCard permettant de créer ces différents objets sont opérationnels.
	\item La sérialisation et la désérialisation du fichier "deck.json" permettant de sauvagarder dans un fichier les modifications faites au deck ( donc, les cartes ajoutées, retirées ou modifiées).
	\item Une gestion basique des erreurs concernant les doublons de cartes, de questions, de decks ainsi que la gestion d'une erreur cernant une question incompatible avec un thème donné lors d'une modification de carte. 
	\item Quelque tests unitaires concernant les méthodes des classes BasicCard, Deck ainsi que Question.
\end{itemize}

Quelques interfaces graphiques ont également déjà été créé. 
\begin{itemize}
	\item Le menu principal permettant de naviguer dans le programme.
	\item Le menu "admin panel" permettant de gérer le jeu tels qu'ajouter une carte ou voire toute les cartes.
	\item L'interface "add a card" permettant d'ajouter une carte une fois tous les textfields remplies.
	\item L'interface permettant de voir les différentes carte créées trié par thème et auteur ( accessible depuis "l'admin panel" dans "show the list of cars").
\end{itemize}
