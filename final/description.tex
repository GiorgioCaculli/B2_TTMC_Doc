\section{Description général de l'application}
L'application inclut beaucoup de fonctionnalitées qui ont été vues au cours.
Les technologies suivantes ont été introduites lors du développement de notre application:
\begin{itemize}
\item le Design Pattern Iterator
\item le Design Pattern Template Method
\item des exceptions customisés permettant de nous donner un message plus claire à un problème rencontré lors d'une quelconque manipulation
\item des tests unitaires afin d'assurer que chaque manipulation se passe correctement
\item la librairie GSON qui nous a facilité la sérialisation et désérialisation des données
\item la librairie JavaFX qui nous a permis de développer un jeu avec un visuel très modérnisé
\item l'utilisation de ResourceBundle afin de nous permettre d'instaurer un système d'internationalisation dans notre programme
\end{itemize}
Ces ne sont qu'une petite partie des technologies que nous avons mis en place pour notre programme.
Afin d'assurer une meilleure portabilité de notre programme pour des gens qui n'utiliserait pas des arguments dans leur \acrshort{vm} de \verb|Java|, ou qui utiliseraient le \acrshort{jdk} de base, on a décidé de mettre en place un \enquote{launcher}.
Ce dernier sert à détécter le système d'exploitation de l'utilisateur, sa version du \acrshort{jdk}, et l'endroit où il est en train de lancer le launcher.
Une fois que toutes ces informations sont chargées, ce launcher va automatiquement lancer le jeu avec les paramètres nécéssaire pour faire fonctionner \verb|JavaFX|.
Cette solutions a énormément simplifié le partage de notre application avec les \enquote{Beta Testers}.
