\section{Présentation du sujet}
nous allons devoir créer un jeu similaire à TTMC.
Il s'agit d'un jeu comportant des cartes comportant des questions. Il existe quatre thèmes repérable par leurs couleurs respectives, la couleur:
\begin{itemize}
	\item mauve est attribué aux cartes ayant pour thème ``improbable''
	\item orange pour ``plaisir''
	\item bleu pour ``informatique''
	\item vert pour ``scolaire''
\end{itemize}
Chaque carte possède un thème, un sujet en rapport avec le thème, et quatre questions.\\
Les question sont numéroté de un à quatre et triées par ordre croissant de difficulté.
\subsection{Règle du jeu}
Le jeu a pour but de démontrer que l'on se connait mieux que les autres se connaissent eux-même.
Pour le montrer, il faut arriver au bout du plateau tout en gagnant des points.
Pour avancer, il faut répondre correctement aux questions qui se poseront, une bonne réponse nous fait avancer d'une case.
Nous pouvons choisir entre quatres difficultés de questions, en fonction de ce que nous connaissons le mieux. Le niveau 1 est le plus facile et le niveau 4 le plus difficile.
A chaque bonne réponse, nous gagnons un nombre de points équivalant au niveau de la question. Par exemple: nous répondons corretement à une question de niveau 1, nous gagnons 1 point. Si nous répondons correctement à une question de niveau 4, 
nous gagnons 4 points. Lorsqu'un joueur arrive sur la dernière case du plateau, il est désigné gagnant si il est le seul à avoir atteint la fin du plateau ou (si ils sont plusieurs sur la dernière case) si il a le plus de points. 

\subsection{Position du problème}
La création d'un jeu vidéo étant une première pour nous, le problème résidait surtout dans la mamière dont nous allions structuré notre projet mais aussi imaginer se jeu autant dans sa présentation graphique que dans les fonctionnalités a 
incorporé. Les fonctionnalités du jeu nous ont demandé beaucoup de réflexions surtout que nous avions de grande ambition.
