\subsection{Launcher}
Le launcher est un logiciel à part entière qui permet à la fois de télécharger JavaFX 11 dépendemment de l'OS utilisé sur le système qui le fait tourner ainsi que de lancer l'application. Le launcher permet donc à n'importe qui de savoir lancer 
notre application tant qu'une version Java 11 au minimum est installé.  
\subsection{Chat en ligne}
Fonctionnalité disponible depuis le jeu en multi joueur en ligne. Chat en ligne permet à plusieurs joueur de communiquer ensemble via un hôte (personne ayant décider d'héberger la partie). Les autres joueur devront rejoindre le jeu de l'hôte. 
Pour le reste cela est un chat dans ce qui a de plus commun, tous les joueurs peuvent envoyer des messages lisible par tous les autres. 
\subsection{Traduction dynamique des menus}
Depuis le menu "setting" (paramètre) il est possible de changer la langue du jeux. En la changeant ainsi qu'en redémarrant le jeu. Le jeu aura changer de langue, tous les boutons des menu seront dès lors traduit dans la langue sélectionné.
Les différentes langues disponible sont 
\begin{itemize}
	\item Anglais (par défaut)
	\item Français
	\item Italien
	\item Japonais
\end{itemize} 