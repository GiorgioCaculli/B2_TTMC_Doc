Différents jeux existent sur le style de \textquote{\acrlong{ttmc}}. Plusieurs nous sont parvenu à la tête:
\begin{itemize}
	\item Trivial Pursuit
	\item Qui veut gagner des millions?
	\item Questions pour un champion
\end{itemize}

Les points intéressants que nous avons relevé du Trivial Pursuit sont les cases \textquote{checkpoint} et les cases à effets particuliers.
Ceux de \textquote{Qui veut gagner des millions ?} sont les effets particuliers que l’on peut appliquer une fois par partie.
Celui de \textquote{Question pour un champion} est principalement le temps maximal pour répondre. 
\newline
En dehors des jeux déjà existants et ressemblants à TTMC, nous avons pensé à différents concepts pouvant améliorer notre jeu.
\begin{itemize}
	\item Des effets aléatoires pouvant affecter les joueurs et/ou leurs points.
	\item Des effets pouvant être gagné en réussissant les cases \textquote{checkpoint}, pouvant cibler soit soi-même ou bien l’adversaire.
	\item Un principe d’ennemi à abattre, affaiblissant celui-ci avec les bonnes réponses que l’on donnerait.
        Cela pourrait aussi permettre de créer un autre mode de jeu, soit le mode versus, où chaque joueur ou équipe affronterait un ennemi et le premier qui le bats à gagner, ou bien un mode coop qui permettrait aux joueurs de s’entraider pour battre cet ennemi.
\end{itemize}
