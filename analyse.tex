Différents jeux existent sur le style de ``\acrlong{ttmc}''. Plusieurs nous sont parvenu à la tête:
\begin{itemize}
	\item Trivial Pursuit
	\item Qui veut gagner des millions?
	\item Questions pour un champion
\end{itemize}
Étant donné que c'est un jeu de table, il ne possède pas de version digitale.\\
Comme pour les exemples cités précédemment, ce jeu implique deux ou plusieurs joueurs. Les participants devront piocher une carte, carte qui sera caractérisé par une couleur, couleur qui caractérisa aussi le thème de la question:
\begin{itemize}
	\item le mauve pour ``improbable''
	\item l'orange pour ``plaisir''
	\item le bleu pour ``informatique''
	\item le vert pour ``scolaire''
\end{itemize}
Lorsque la carte sera pioché, si la réponse est correcte, les points correspondants à la difficulté de la carte lui seront attribué, si la réponse est erronée, alors le jouer ne gagnera aucun point.
