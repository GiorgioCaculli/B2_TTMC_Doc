Code les plus représentatifs du Sprint 2
\begin{lstlisting}
  public void setVisibleNode( String paneName )
  {

    for ( Node n : getStackPane().getChildren() )
    {
      if ( n.getClass().getSimpleName().equals( paneName ) )
      {
        n.setVisible( true );
      }
      else
      {
        n.setVisible( false );
      }
    }
  }
  

  getStackPane().getChildren().add( new MenuPrincipalBP( d ) );
  getStackPane().getChildren().add( new MenuPlayBP( d ) );
  getStackPane().getChildren().add( new MenuAdminBP( d ) );

  setVisibleNode( MenuPrincipalBP.class.getSimpleName() );
\end{lstlisting}

Ce code permet de choisir quel Node le StackPane devrait rendre visible sur base d'une chaîne de caractères.
Tous les autres Nodes seront cachés si le nom de la classe ne correspond pas à celle que l'on souhaite rendre visible.
Indépendamment de l'ordre, la fonction sera capable de rendre un seul Node visible.

Ici par exemple, un StackPane contient 3 Nodes:
\begin{itemize}
\item MenuPrincipalBP
\item MenuPlayBP
\item MenuAdminBP
\end{itemize}

Lors de l'appel de la fonction setVisibleNode, on prendra en paramètre une chaîne de caractères, soit le nom de la classe que l'on souhaite afficher.

On peut obtenir le nom de la classe sans faire aucune faute de frappe, en appelant la fonction propre à chaque classe: \verb|getSimpleName()|.
Grâce à cette fonction, peut importe la casse, la chaîne de caractères introduite sera toujours le nom du Node que l'on souhaite montrer.
