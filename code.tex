Code les plus représentatifs du Sprint 2
\begin{lstlisting}
  public void setVisibleNode( String paneName )
  {

    for ( Node n : getStackPane().getChildren() )
    {
      if ( n.getClass().getSimpleName().equals( paneName ) )
      {
        n.setVisible( true );
      }
      else
      {
        n.setVisible( false );
      }
    }
  }
  

  getStackPane().getChildren().add( new MenuPrincipalBP( d ) );
  getStackPane().getChildren().add( new MenuPlayBP( d ) );
  getStackPane().getChildren().add( new MenuAdminBP( d ) );

  setVisibleNode( MenuPrincipalBP.class.getSimpleName() );
\end{lstlisting}

Ce code permet de choisir quel Node le stackpane devrait rendre visible sur base d'une chaîne de caractère.
Tous les autres Node seront cachés si le nom de la classe ne corréspond pas à celle qu'on souhaite visible.
Indépendammente de l'ordre, la fonction sera capabe de rendre un seul Pane visible.

Ici par exemple, une StackPane contient 3 Node:
\begin{itemize}
\item MenuPrincipalBP
\item MenuPlayBP
\item MenuAdminBP
\end{itemize}

Lors de l'appel de la fonction setVisibleNode, on prendra en paramètre une chaîne de caractère, soit le nom de la classe que l'on souhaite charger.

On peut obtenir le nom de la classe sans faire aucune faute de frappe, en appelant fonction propre à chaque classe: \verb|getSimpleName()|.
Grâce à cette fonction, peut importe la casse, le paneName introduit sera toujours celui du Node qu'on souhaite montrer.
